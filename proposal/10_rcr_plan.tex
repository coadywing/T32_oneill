% 10_rcr_plan.tex — Plan for Instruction in Responsible Conduct of Research
% (3 pages, separate attachment)
%
% Now scored as part of overall impact.
%
% NIH minimum standards:
% - Format: must include face-to-face discussion (online-only unacceptable)
% - Duration: minimum 8 contact hours
% - Frequency: at least once per career stage, no less than every 4 years
% - Subject matter: conflict of interest, authorship, data management,
%   human subjects, animal use, lab safety, research misconduct, ethics
% - Faculty participation: must be described

\section*{Plan for Instruction in the Responsible Conduct of Research}

All trainees will complete formal instruction in the Responsible Conduct of Research
per NIH requirements. Training will include:

\begin{enumerate}
  \item \textbf{Face-to-face discussion sessions} led by program faculty, covering
    ethical case studies drawn from health economics and aging research;
  \item \textbf{Formal coursework} covering data acquisition and management,
    responsible authorship, peer review, mentor-trainee responsibilities, conflicts of
    interest, research misconduct, and human subjects research;
  \item \textbf{Case study discussions} integrated into the weekly seminar series,
    addressing ethical challenges specific to research with older adult populations and
    administrative health data.
\end{enumerate}

Training will be completed in the first year of appointment and refreshed annually.

% TODO: Expand substantially to fill 3 pages. Required details:
%
% FORMAT AND SCHEDULE
%   - Specify the exact format: workshops, seminars, small-group discussions
%   - Calendar of RCR activities (e.g., 4 two-hour sessions in fall semester)
%   - Total contact hours (must be ≥8; aim for 12-16)
%   - Refresher schedule for continuing trainees
%
% SUBJECT MATTER (must cover all of these)
%   - Conflict of interest — financial and non-financial
%   - Responsible authorship and publication practices
%   - Data acquisition, management, sharing, and ownership
%   - Human subjects protections (especially relevant for Medicare data)
%   - Animal welfare (brief — note that program is not animal-based)
%   - Laboratory safety (adapt to computational research context)
%   - Research misconduct — fabrication, falsification, plagiarism
%   - Research ethics — broader ethical considerations
%   - Mentor-trainee responsibilities
%   - Peer review
%   - Collaborative research
%   - Rigor and reproducibility (may overlap with separate attachment)
%
% FACULTY PARTICIPATION
%   - Name specific faculty who will lead RCR sessions
%   - Describe their qualifications for teaching RCR
%   - Guest speakers (e.g., IU Research Ethics Officer, IRB staff)
%
% SPECIAL CONSIDERATIONS FOR THIS PROGRAM
%   - Ethical use of administrative health data (Medicare claims, Medicaid)
%   - Data security and privacy (HIPAA, CMS data use agreements)
%   - Responsible use of linked datasets
%   - Ethical considerations in policy-relevant research
%   - Publication ethics in economics (pre-registration, replication)
%
% ASSESSMENT
%   - How will trainee understanding of RCR be assessed?
%   - Written reflections, case study analyses, or other deliverables?
