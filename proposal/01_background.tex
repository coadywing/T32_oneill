% 01_background.tex — Background (opens the 25-page Research Training Program Plan)
%
% NOTE: T32s do NOT have a separate Specific Aims page (that's an R01 convention).
% Instead, the program overview, goals, and objectives open the Research Training
% Program Plan. This section serves that purpose.

\section{Background}

\subsection{Program Overview}

The O'Neill School of Public and Environmental Affairs at Indiana University Bloomington
proposes an NIA T32 Institutional Training Grant to prepare predoctoral students for
careers advancing research on health economics and policy affecting older adults.
The rapid aging of the U.S.\ population---with Medicare enrollment projected to exceed
80 million by 2030 and Alzheimer's disease prevalence expected to nearly triple by
2060---demands a robust pipeline of researchers trained to evaluate the effectiveness,
equity, and efficiency of programs serving older adults. Yet few doctoral training
programs integrate rigorous econometric methods with deep expertise in aging policy.

% TODO: Sharpen the gap statement — what specifically is missing in the current
% training landscape that this program addresses?

Building on our established strength in health economics doctoral training and our
newly funded NIA P01 Program Project Grant on Alzheimer's Disease and Related Dementias
(ADRD; \$15.9M, 2024--2029), this program will train the next generation of researchers
to address critical questions about healthcare delivery, health insurance, and policy
interventions for aging populations. We request 3 predoctoral training slots annually
for a 5-year project period.

\subsection{Program Goals and Objectives}

The program's primary objectives are to train predoctoral scholars who will:

\begin{enumerate}
  \item \textbf{Apply rigorous econometric methods} to study health policy impacts on
    older adults, including causal identification strategies such as
    difference-in-differences, instrumental variables, and regression discontinuity
    designs;
  \item \textbf{Develop expertise in analyzing large administrative datasets} relevant
    to aging populations, including Medicare claims, long-term care assessment data, and
    linked survey-administrative data;
  \item \textbf{Conduct policy-relevant research} on Medicare and Medicaid programs,
    long-term care, dementia caregiving, and healthcare access for older adults, with
    direct engagement in P01 research on ADRD healthcare delivery;
  \item \textbf{Translate research findings into actionable policy recommendations}
    through engagement with policymakers and stakeholders, professional development in
    grant writing and scientific communication, and exposure to diverse career pathways
    in aging research.
\end{enumerate}

Over the past 15+ years, the core faculty have mentored over 20 PhD trainees to
completion, with 82\% of recent graduates holding tenure-track academic positions at
research universities. This training program will expand the pipeline of health
economists prepared to address the research challenges posed by an aging society.

\subsection{Alignment with NIA Mission}

This training program directly supports NIA's mission to understand the nature of aging
and extend healthy, active years of life. Our focus on health economics and policy
research addresses NIA's strategic goal of improving the health, well-being, and
independence of older adults through research that informs Medicare policy, evaluates
interventions to improve care quality, and examines the economic dimensions of ADRD
caregiving and management of other chronic conditions.

% TODO: Add specific citations to NIA strategic plan goals
% TODO: Reference specific NIA priority areas (e.g., AD/ADRD National Plan milestones)

\subsection{Landscape of NIA T32 Programs}

The proposed program would complement existing NIA-funded T32 programs while filling a
unique niche in health economics and policy training with an applied microeconomics
foundation.

\begin{tabularx}{\textwidth}{lXX}
\toprule
\textbf{Program} & \textbf{Focus} & \textbf{O'Neill Differentiation} \\
\midrule
NBER Aging \& Health Economics (T32AG000186) &
  Health economics; 150+ trainees since 1989; consortium model with multiple
  universities &
  O'Neill offers intensive mentorship in single institution; strong public policy
  orientation \\
U.\ Chicago Demography \& Economics of Aging (T32AG000243) &
  Demography; economics of aging; Harris School of Public Policy connection &
  O'Neill emphasizes health services research and Medicare/Medicaid policy
  evaluation \\
Johns Hopkins Health Services for Aging (T32) &
  Health services research; clinical focus; nursing and medical school
  integration &
  O'Neill provides economics-grounded causal inference training; policy analysis
  emphasis \\
UCLA/RAND Health Services Research (AHRQ T32) &
  Health policy and management; RAND collaboration; quality and efficiency focus &
  O'Neill uniquely combines \#1-ranked public affairs with NIA P01
  infrastructure \\
\bottomrule
\end{tabularx}

The O'Neill T32 offers unique advantages: (1) Integration with an active NIA P01
providing immediate research engagement on ADRD; (2) Training in a \#1-ranked public
affairs program emphasizing policy translation; (3) Access to Regenstrief Institute's
clinical informatics resources for future Medicare-EHR linkage; (4) Faculty with NBER
affiliations providing national network access; (5) Strong track record of placing
graduates in aging-relevant academic and policy positions.
