% 05_program_evaluation.tex — Program Evaluation (part of 25-page Program Plan)

\section{Program Evaluation}

\subsection{Evaluation Metrics}

% TODO: The draft references historical metrics (82% tenure-track placement) as
% targets. Need to formalize these into a structured evaluation framework.
% The FOA asks for:
%   - Assessment of program effectiveness in meeting goals/objectives
%   - Procedures for responding to evaluation findings

Historical performance data support the following targets:

\begin{tabularx}{\textwidth}{lXl}
\toprule
\textbf{Outcome Domain} & \textbf{Metrics} & \textbf{Target} \\
\midrule
Training Completion &
  PhD completion rate; time to degree &
  $>$90\% within 6 years \\
Research Productivity &
  Peer-reviewed publications; conference presentations &
  2+ publications per trainee \\
Grant Success &
  F31 applications submitted; funded &
  $>$50\% submit F31; $>$25\% funded \\
Career Placement &
  Academic positions; research-intensive careers &
  $>$80\% in research positions within 1 year \\
Aging Research Focus &
  Dissertations on aging topics; continued aging research &
  $>$75\% with aging-focused dissertations \\
\bottomrule
\end{tabularx}

% TODO: Add data collection plan — how will these metrics be tracked?
% TODO: Add formative assessment metrics (annual milestones, coursework progress,
%   qualifying exam passage rates)

\subsection{Continuous Improvement}

The program will conduct annual reviews including: trainee progress assessments,
mentor-trainee meeting evaluations, curriculum review based on trainee and faculty
feedback, and external advisory board input. The Program Director will prepare annual
reports tracking outcomes against targets and implementing improvements based on
evaluation findings.

% TODO: Expand this section substantially:
%   - Describe the annual review process in detail
%   - How are trainee evaluations conducted? (anonymous surveys, exit interviews)
%   - What triggers a curriculum change?
%   - How is feedback from current and former trainees collected?
%   - Role of the advisory committee in evaluation
%   - Describe how evaluation findings feed back into program modifications
