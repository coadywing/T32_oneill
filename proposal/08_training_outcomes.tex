% 08_training_outcomes.tex — Training Outcomes (part of 25-page Program Plan)

\section{Training Outcomes}

\subsection{Evidence of Training Excellence}

The training faculty have an exceptional and sustained record of doctoral mentoring
spanning more than 15 years. Across this period, the core faculty (Simon, Wing,
Freedman) have collectively mentored over 20 PhD trainees to completion, with the
majority now holding tenure-track academic positions or senior research roles in
government and policy organizations directly relevant to the NIA mission.

Alumni placements include R1 research universities (Georgia Tech, USC Price, Penn
State, University of Washington, Arizona State, Purdue, Florida International),
federal policy institutions (U.S.\ GAO), and applied research organizations focused
on health and aging (American Institutes for Research, Institute for Research on
Poverty). Many alumni conduct research directly relevant to aging populations,
including Medicare and Medicaid policy, health insurance design, long-term care
financing, dementia caregiving, and social safety net programs affecting older adults.

\subsection{Recent PhD Placements (2023--2025)}

\begin{tabularx}{\textwidth}{llXl}
\toprule
\textbf{Trainee} & \textbf{Year} & \textbf{Current Position} & \textbf{Committee} \\
\midrule
Megdalynn Fisher     & 2025 & Asst.\ Professor, Florida International University          & Simon, Wing \\
Dario Salcedo Monroy & 2025 & Research Associate, Inst.\ for Research on Poverty (UW-Madison) & Simon, Wing, Freedman \\
Luis Navarro         & 2025 & Asst.\ Professor, U.\ Washington Evans School               & Wing \\
Lanjun Peng          & 2025 & Asst.\ Professor, Arizona State University                  & Wing \\
Madeline Yozwiak     & 2025 & Postdoctoral Fellow, Carnegie Mellon University              & Simon, Wing \\
Marylis Fantoni      & 2025 & Asst.\ Professor, Brigham Young University                  & Wing \\
Ashley C.\ Bradford  & 2023 & Asst.\ Professor, Georgia Tech (Public Policy)              & Simon, Wing, Freedman \\
Patrick Carlin       & 2023 & Asst.\ Professor, Washington State University               & Simon, Wing \\
Laura Montenovo      & 2023 & Asst.\ Professor, Purdue University                         & Simon, Wing, Freedman \\
\bottomrule
\end{tabularx}

\subsection{Earlier PhD Placements (2017--2021)}

\begin{tabularx}{\textwidth}{llXl}
\toprule
\textbf{Trainee} & \textbf{Year} & \textbf{Current Position} & \textbf{Committee} \\
\midrule
Michelle Graff       & 2021 & Asst.\ Professor, Georgia Tech (Carter School)    & Freedman \\
Patrick F.\ Hibbard  & 2021 & Research Scientist, Chestnut Health Systems        & Wing \\
Johabed G.\ Olvera   & 2020 & Asst.\ Professor, Penn State (Public Policy)      & Simon, Wing, Freedman \\
Joanna Carroll       & 2020 & Economist, U.S.\ Government Accountability Office & Simon, Wing \\
Ngoc Dao             & 2019 & Asst.\ Professor, Kean University                 & Simon, Freedman \\
Lindsey Bullinger    & 2018 & Assoc.\ Professor, Georgia Tech; Co-Editor \textit{JPAM} & Simon, Wing \\
Noah Hammarlund      & 2018 & Asst.\ Professor, U.\ Florida Health Services Research    & Simon, Wing, Freedman \\
Shun-Wen Wu          & 2017 & Asst.\ Professor, National Taiwan University       & Simon, Wing \\
\bottomrule
\end{tabularx}

\subsection{Placement Summary}

Of the 17 trainees with documented placements, 14 hold tenure-track or tenured academic
positions at research universities, 1 is in federal government policy analysis (GAO),
1 is in applied health research (Chestnut Health Systems), and 1 is in a competitive
postdoctoral fellowship (Carnegie Mellon). This 82\% academic placement rate is notably
high for doctoral programs in public policy and health economics.

\subsection{Alumni with Aging-Relevant Research Portfolios}

Several former trainees have developed substantial research portfolios directly aligned
with NIA's mission in aging and health policy:

\begin{tabularx}{\textwidth}{lXX}
\toprule
\textbf{Trainee} & \textbf{Aging-Relevant Research} & \textbf{Impact Indicators} \\
\midrule
Ashley Bradford (Georgia Tech) &
  Medicare Part D prescription patterns; opioid prescribing in Medicare;
  buprenorphine access for Medicare beneficiaries with OUD &
  \textit{JAMA Internal Medicine}, \textit{Health Affairs} (Most Read/Shared);
  media: NYT, Washington Post, CNN, BBC \\
Dan Shane (U.\ Iowa) &
  Medicare Part D effects on mental health; Medicare Part D and ED utilization;
  rural health access for aging populations &
  \textit{Journal of Health Economics}, \textit{Health Economics}; teaches graduate
  course on Medicare/Medicaid policy \\
Noah Hammarlund (U.\ Florida) &
  Rural-urban differences in modifiable dementia risk factors; AI/ML applications
  to cancer care disparities in older adults &
  NLM informatics postdoctoral fellowship; UF Health Cancer Center pilot grant;
  health equity focus \\
Ruth Winecoff (Rutgers) &
  ACA Medicaid expansion effects on near-elderly adults (ages 50--64) using Health
  and Retirement Study data &
  \textit{INQUIRY} publication with K.\ Simon examining health outcomes for
  near-elderly population \\
Angshuman Gooptu (AIR) &
  ``Medicaid and the Elderly'' (Federal Reserve Bank of Chicago); pathways to
  Medicaid coverage for adults 65+ &
  \textit{Health Affairs}; CMS and HRSA federal research projects; direct
  policy impact \\
\bottomrule
\end{tabularx}

\subsection{Training Record Summary}

This training record demonstrates three key strengths relevant to NIA T32 review
criteria:

\begin{enumerate}
  \item \textbf{Sustained mentoring capacity} --- the same core faculty have
    successfully mentored multiple cohorts over 15+ years, indicating a true training
    program rather than isolated individual successes;
  \item \textbf{High-quality placements aligned with NIA priorities} --- trainees
    consistently place in tenure-track positions at research universities and in
    policy-relevant government and research organizations where they can advance aging
    research;
  \item \textbf{Research continuity in aging} --- multiple alumni have developed
    independent research programs directly addressing Medicare/Medicaid policy, dementia,
    long-term care, and health services for older adults.
\end{enumerate}

The committee membership patterns in the tables above further demonstrate team-based,
interdisciplinary mentoring rather than single-advisor training.
