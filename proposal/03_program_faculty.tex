% 03_program_faculty.tex — Program Faculty (part of 25-page Program Plan)

\section{Program Faculty}

\subsection{Core Training Faculty}

Our training faculty includes nationally recognized health economists with extensive
experience mentoring doctoral students and strong track records in aging-related research.

\begin{tabularx}{\textwidth}{lXX}
\toprule
\textbf{Faculty} & \textbf{Expertise \& Aging Research} & \textbf{Training Record} \\
\midrule
Kosali Simon, PhD --- Distinguished Professor, Program Director &
  Health insurance policy; Medicare/Medicaid; ADRD healthcare delivery; MPI on NIA P01
  (\$15.9M); Editor, \textit{Journal of Health Economics}; Co-Editor, \textit{Journal of
  Human Resources}; NBER Research Associate &
  17+ PhD students on dissertation committees; former trainees placed at Georgia Tech,
  Purdue, UGA, U.\ Florida, U.\ Iowa, federal agencies (GAO, AIR) \\
Coady Wing, PhD --- Professor &
  Causal inference methods; health policy evaluation; econometric methods for
  quasi-experimental designs; opioid policy research &
  20+ PhD students on dissertation committees; expertise in methods training; students
  placed at Georgia Tech, Purdue, U.\ Washington, Northwestern, federal agencies \\
Seth Freedman, PhD --- Associate Professor &
  Health economics; hospital markets; healthcare competition; Medicare payment policy;
  health services utilization &
  12+ PhD students on dissertation committees; expertise in Medicare research methods;
  students placed at U.\ Michigan, U.\ Wisconsin, policy research organizations \\
\bottomrule
\end{tabularx}

\subsection{Affiliated Faculty with Aging Expertise}

Additional training resources include affiliated faculty from IU School of Medicine
(geriatrics, internal medicine), Regenstrief Institute (health informatics, clinical
research), and IU Kelley School of Business (health economics). These collaborations
provide trainees with interdisciplinary perspectives essential for impactful aging
research.

% TODO: Name specific affiliated faculty with their expertise and role in the program
% TODO: Describe how affiliated faculty interact with trainees (guest lectures,
%   committee service, co-mentoring, seminar participation)

\subsection{Mentor Training Strategy}

% TODO: This is a key required element. The FOA specifically asks for:
%   - Mentor training format, duration, and frequency
%   - Major topics: aligning expectations, effective communication, fostering
%     independence, assessing understanding, professional development,
%     articulating mentoring philosophy
%   - Mechanism for monitoring mentoring effectiveness
%   - Process for removing underperforming faculty
%
% This section needs substantial development. Consider:
%   - Annual mentor training workshops (e.g., CIMER Entering Mentoring curriculum)
%   - Mentoring compacts between faculty and trainees
%   - Annual trainee evaluations of mentoring quality
%   - Advisory committee review of mentoring effectiveness
%   - Clear procedures if a mentor is not meeting expectations

\textit{[Mentor training strategy to be written. This is a heavily weighted review
element --- reviewers assess whether there is a ``strong plan for faculty participation
in evidence-informed mentoring practices that promote the development of all
trainees.'']}
