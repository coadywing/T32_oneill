% 07_institutional_environment.tex — Institutional Environment and Commitment
% (part of 25-page Program Plan)

\section{Institutional Environment and Commitment}

\subsection{O'Neill School Excellence}

The O'Neill School is ranked \#1 among public affairs programs nationally (U.S.\ News
\& World Report), with Health Policy and Management ranked \#5. The School has a
distinguished 50+ year history of training doctoral students who become leaders in
academia, government, and policy research. Our PhD program in Public Affairs with a
health economics concentration provides rigorous training in microeconomic theory,
econometrics, and applied health policy analysis.

\subsection{NIA P01 Program Project: Foundation for Training}

In 2024, NIA funded a \$15.9 million Program Project Grant (P01) led by Dr.\ Kosali
Simon (MPI) examining healthcare delivery for persons with Alzheimer's Disease and
Related Dementias. This five-year project (2024--2029) at NBER provides an exceptional
platform for doctoral training, offering students opportunities to engage with
cutting-edge ADRD research, potential access to linked Medicare-clinical data through
partnership with Regenstrief Institute, and mentorship from a multidisciplinary team
spanning health economics, geriatric medicine, and health services research.

% TODO: Clarify the NBER/IU relationship — is the P01 administered through NBER
%   with IU subcontract, or vice versa? How does this affect trainee access?

\subsection{Strategic Partnerships}

\begin{tabularx}{\textwidth}{lX}
\toprule
\textbf{Partner} & \textbf{Training Resources} \\
\midrule
Regenstrief Institute &
  Access to Indiana Network for Patient Care (INPC) clinical data;
  collaboration with health informatics researchers; data linkage expertise
  for future Medicare-EHR research \\
IU School of Medicine &
  Geriatrics faculty collaboration; Indiana Alzheimer's Disease Research
  Center; clinical perspectives on aging research \\
NBER Health Economics Program &
  Faculty affiliations with NBER; Summer Institute participation; access to
  leading health economists nationally \\
CMS Data Access &
  Established Data Use Agreements for Medicare Limited Data Set;
  ResDAC-approved protocols for claims analysis \\
\bottomrule
\end{tabularx}

\subsection{Institutional Commitment}

Indiana University and the O'Neill School commit to:

\begin{enumerate}
  \item Cost-sharing for tuition beyond NIH allowances;
  \item Office and computing resources for all trainees;
  \item Travel supplements for additional conference attendance;
  \item Administrative support for program coordination;
  \item Research infrastructure including secure data environments for Medicare/Medicaid
    analysis.
\end{enumerate}

% TODO: Quantify the commitment where possible:
%   - Dollar amounts for tuition cost-sharing
%   - Square footage of trainee office space
%   - Description of computing infrastructure (HPC clusters, secure data rooms)
%   - Named administrative staff supporting the program
%   - Protected time for faculty mentoring (teaching releases?)

\subsection{Safe and Supportive Research Environment}

% TODO: Required by FOA — describe:
%   - IU policies preventing discrimination and harassment
%   - Reporting mechanisms and accountability procedures
%   - Support services for trainees
%   - Institutional commitment to diversity and inclusion
%   - Reference to institutional letter of support (separate attachment, max 10 pages)

\textit{[Safe environment description to be written. The institutional letter of support
(max 10 pages, from Provost or Dean) should describe these commitments in detail.]}

\subsection{Distinction from Other Training Programs}

% TODO: FOA requires this for institutions with multiple NIH training grants.
%   - List other NIH-funded training programs at IU
%   - Explain how this T32 is distinct
%   - Describe any synergies or resource-sharing

\textit{[Distinction from other IU training programs to be described. Check NIH
Reporter for current T32s at IU Bloomington.]}
