% 02_program_admin.tex — Program Administration (part of 25-page Program Plan)

\section{Program Administration}

\subsection{Multiple Program Director Leadership Structure}

This T32 employs a Multiple Program Director (PD/PI) structure to reflect the
interdisciplinary and team-based nature of training in health economics and aging
policy. The three PDs represent the tenured health economics faculty at O'Neill and
provide complementary expertise spanning health insurance policy, causal inference
methods, and Medicare payment and delivery systems. This leadership structure ensures
continuity, depth of mentoring capacity, and sustained program oversight across multiple
faculty careers.

\begin{tabularx}{\textwidth}{llX}
\toprule
\textbf{Program Director} & \textbf{Role} & \textbf{Responsibilities} \\
\midrule
Kosali Simon, PhD (Contact PD) & Overall Leadership &
  NIA interface; program evaluation and reporting; institutional coordination;
  P01 integration oversight \\
Coady Wing, PhD & Curriculum \& Methods &
  Econometric methods training; curriculum design and assessment; seminar
  coordination; research methods workshops \\
Seth Freedman, PhD & Recruitment \& Mentoring &
  Trainee recruitment and selection; mentoring oversight; professional
  development; job market preparation \\
\bottomrule
\end{tabularx}

% TODO: Specify percent effort for each PD/PI (FOA requires this)
% TODO: Document previous mentoring experience for each PD (number of students,
%   placements, years of mentoring)

\subsection{Physical Co-Location and Mentoring Intensity}

The three Program Directors and trainees are physically co-located on a single floor
within the O'Neill School, with doctoral student and postdoctoral offices embedded
within the same suite as faculty offices. This arrangement facilitates daily informal
interaction, rapid feedback on research, and close mentoring beyond scheduled meetings.
Trainees have immediate access to all three PDs for methodological questions, research
guidance, and professional development conversations. This intensive co-located
environment fosters a cohesive training community, enables early identification of
trainee needs, and creates natural opportunities for collaborative research across
faculty projects.

\subsection{Mentor Training Plan}

% TODO: This is a required element — the FOA asks for:
%   - Plan for PD/PI to receive appropriate mentor training
%   - Format, duration, and frequency of mentor training
%   - Evidence that PDs have completed or will complete formal mentor training
%     prior to program start
% Consider: CIMER mentor training, IU Center for Innovative Teaching and Learning
% programs, National Research Mentoring Network (NRMN) resources

\textit{[Mentor training plan to be written. The FOA requires documentation that PDs
have completed or will complete formal mentor training prior to program start.]}

\subsection{Advisory Committee}

% TODO: Describe external advisory committee structure
% Consider: composition (external scholars, NIA-funded investigators, policy
% stakeholders), meeting frequency, role in program evaluation

\textit{[Advisory committee plan to be written. NIH highly recommends including an
advisory committee plan filed as ``Advisory\_Committee.pdf.'']}
