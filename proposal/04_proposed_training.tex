% 04_proposed_training.tex — Proposed Training (part of 25-page Program Plan)

\section{Proposed Training}

\subsection{Core Curriculum}

Trainees complete rigorous coursework in three domains:

\subsubsection{Microeconomic Theory and Econometrics}

PhD-level microeconomics sequence; advanced econometrics including panel data methods,
instrumental variables, difference-in-differences, regression discontinuity; Bayesian
methods.

% TODO: List specific course numbers and titles
% TODO: Describe the sequencing — what do students take in Year 1 vs. Year 2+?

\subsubsection{Health Economics and Policy}

H354 Health Economics; healthcare markets; health insurance economics; pharmaceutical
economics; behavioral health economics.

% TODO: List specific courses with numbers
% TODO: Describe how health economics training integrates with the broader
%   O'Neill PhD curriculum

\subsubsection{Aging-Specific Training}

Economics of aging seminar; Medicare/Medicaid policy analysis; dementia care economics;
long-term care financing; NIA P01 research seminars.

% TODO: Is there a formal aging seminar course, or is this the P01 seminar series?
% TODO: Describe the P01 research seminar in more detail — frequency, format,
%   how trainees participate

\subsection{Research Training Components}

\begin{tabularx}{\textwidth}{lX}
\toprule
\textbf{Component} & \textbf{Description} \\
\midrule
Mentored Research &
  Each trainee works closely with a primary mentor on an aging-focused research project.
  Year 1: literature review and data access. Year 2: analysis and manuscript preparation.
  Trainees are expected to submit at least one first-authored paper to a peer-reviewed
  journal. \\
P01 Integration &
  Trainees participate in NIA P01 research meetings (biweekly), gaining exposure to ADRD
  research across multiple projects. Opportunities to contribute to P01 analyses as
  co-investigators. \\
Data Training &
  Hands-on training in Medicare claims analysis; MEPS; HRS; state Medicaid data. Emphasis
  on reproducible research workflows (version control, documentation, code review). \\
Seminar Series &
  Weekly health policy seminar (in 14th year currently) featuring external speakers;
  monthly ADRD research seminar; annual NBER connections through faculty affiliations. \\
Grant Writing &
  Training in NIH grant mechanisms (F31, R03, R21, R01); mock study sections; preparation
  of F31 applications with faculty mentorship. \\
Professional Development &
  Presentation skills; conference participation (ASHEcon, APPAM, AEA); job market
  preparation; career mentoring for academic and policy careers. \\
\bottomrule
\end{tabularx}

\subsection{Career Development}

% TODO: The FOA requires substantial content on career development:
%   - Information on biomedical research workforce employment landscape
%   - Range of career paths for which training is useful
%   - Career outcomes of program graduates (must be publicly accessible)
%   - Engagement of potential employers
%   - Experiential learning opportunities (internships, shadowing, teaching,
%     informational interviews)
%
% The table above covers some of this (grant writing, conferences, job market).
% Expand into a narrative describing:
%   - The landscape of careers for health economists in aging
%   - How the program prepares for both academic and policy careers
%   - Alumni network and employer engagement
%   - Teaching opportunities

\textit{[Career development narrative to be expanded. The table above covers key
components; this subsection should provide context on the employment landscape and
how program activities connect to career outcomes.]}

\subsection{Individual Development Plans and Data Training}

Each trainee will develop and maintain an Individual Development Plan (IDP) in
consultation with their primary mentor, updated annually to track progress toward
career goals. Trainees will also receive training in data transparency, sharing, and
storage practices consistent with NIH data management and sharing policies, including
training on secure handling of Medicare and other administrative health data.
