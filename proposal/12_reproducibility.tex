% 12_reproducibility.tex — Plans for Instruction in Methods for Enhancing Reproducibility
% (3 pages, separate attachment)
%
% This is a REQUIRED separate attachment for T32 applications.
% Reviewers assess whether trainees receive training in:
%   - Scientific reasoning and rigorous research design
%   - Experimental methods and relevant biological variables
%   - Resource authentication
%   - Quantitative approaches and data analysis/interpretation

\section*{Plans for Instruction in Methods for Enhancing Reproducibility}

% TODO: Write the full reproducibility plan. This is distinct from RCR.
%
% KEY ELEMENTS TO COVER:
%
% 1. RIGOROUS RESEARCH DESIGN
%   - Training in causal inference methods (core to our program)
%   - Identification strategies and their assumptions
%   - Power analysis and sample size considerations
%   - Pre-analysis plans and pre-registration
%   - Threats to internal and external validity
%
% 2. QUANTITATIVE APPROACHES
%   - Advanced econometrics coursework
%   - Sensitivity analysis and robustness checks
%   - Multiple testing corrections
%   - Replication as a pedagogical tool
%   - Computational reproducibility (code, data, documentation)
%
% 3. DATA MANAGEMENT AND TRANSPARENCY
%   - Reproducible research workflows (version control, literate programming)
%   - Data documentation and codebooks
%   - NIH data management and sharing policies
%   - Secure handling of administrative health data
%   - Code sharing and archiving practices
%
% 4. RELEVANT BIOLOGICAL/CLINICAL VARIABLES
%   - For health economics: understanding how clinical measurement affects
%     economic analysis (e.g., diagnosis coding in claims data, measurement
%     of cognitive decline in ADRD research)
%   - Sex/gender as a biological variable in health economics research
%   - Race/ethnicity measurement and its implications for policy analysis
%   - Comorbidity measurement and risk adjustment
%
% 5. HOW REPRODUCIBILITY TRAINING IS DELIVERED
%   - Integrated into econometrics coursework
%   - Replication exercises in research seminars
%   - Code review practices in the research group
%   - Requirement to maintain reproducible research archives
%   - P01 project standards as a model
%
% 6. ASSESSMENT
%   - How will proficiency be evaluated?
%   - Replication exercises, reproducible dissertation chapters

\textit{[Reproducibility plan to be written. This is a separate 3-page attachment.
For a health economics program, this should emphasize causal inference methodology,
computational reproducibility, and transparent data practices rather than wet-lab
reproducibility concerns.]}
